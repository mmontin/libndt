% Imported packages
%\usepackage[utf8x]{inputenc}
\usepackage{amsmath}
\usepackage{amsfonts}
\usepackage{amssymb}
\usepackage[svgnames]{xcolor} % Provides better colors with upper cases
\usepackage{agda} % The Agda package to input Agda code
%\usepackage[right]{lineno} % Allows to number lines
\setlength\linenumbersep{-8pt} % Moves left by 8pt the line numbers
\usepackage{xargs} % Allows to create better commands
\usepackage{tikz}
\usepackage{xspace}
%\usepackage{graphicx}
\usetikzlibrary{positioning,backgrounds,arrows,automata}
\usepackage{multicol}
\usepackage{mdframed}

% Comments macros
\newcommand{\cath}[1]{\textcolor{Blue}{CD : #1}}
\newcommand{\amelie}[1]{\textcolor{Red}{AL : #1}}
\newcommand{\mathieu}[1]{\textcolor{Green}{MM : #1}}

% Special words macros
\newcommand{\coq}{\textsc{Coq}\xspace}
\newcommand{\agda}{\textsc{Agda}\xspace}
\newcommand{\lagda}{\texttt{lagda}\xspace}
\newcommand{\haskell}{\textsc{Haskell}\xspace}
\newcommand{\datatypes}{datatypes\xspace}
\newcommand{\Datatypes}{Datatypes\xspace}
\newcommand{\datatype}{datatype\xspace}
\newcommand{\libName}{\textsc{LibNDT}\xspace}
\newcommand{\linear}{\textsc{LNDT}\xspace}
\newcommand{\linears}{\textsc{LNDTs}\xspace}
\newcommand{\linearsFull}{linked nested \datatypes\xspace}

% URL macros
\newcommand{\gitURL}{\url{https://bitbucket.org/MonsieurO/nested/}}

% Mapping of unknown characters
\DeclareUnicodeCharacter{8759}{\ensuremath{::}}
\DeclareUnicodeCharacter{955}{\ensuremath{\lambda}}
\DeclareUnicodeCharacter{969}{\ensuremath{\omega}}

% Other macros
\AgdaNoSpaceAroundCode % The name says it all
\renewcommandx*{\hrulefill}[2][1=0.3mm,2=0pt] % A \rule with parameterized positioning
    {\leavevmode \leaders \hbox to 1pt{\rule[#2]{1pt}{#1}} \hfill \kern 0pt}

\setlength{\mathindent}{.8 cm}%

% Macros to define paragraphs, either the first of the section, or the next one
% This is useful because lipics treats paragraphs like sections, which is not always what we want
\newcommand{\myfirstparagraph}[1]{\noindent\textbf{#1} --}%
\newcommand{\myparagraph}[1]{\vspace{.2cm}\myfirstparagraph{#1}}%

\tikzset{
    node distance=5mm and -9mm, %Caractéristiques sur les noeuds
    boite/.style={ %Caractéristiques sur le styles des boites
        fill=blue!50!cyan!70,
        draw,
        align=center,
        font=\footnotesize\bfseries,
        text=white,
    },
    boite coins ronds/.style={ %Caractéristiques sur les boites
        boite,
        rounded corners=3pt
    },
    boite circulaire/.style={ %Caractéristiques sur les boites circulaires
        boite,
        circle
    },
    fleche/.style={ %Caractéristiques sur les flèches
        line cap=round,
        -latex,
        line width=0.25mm,
	    draw=blue!50!cyan!30,
	},
	grosse fleche/.style={ %Caractéristiques sur les grosses flèches
	    fleche,
	    line width=1mm
	},
}
